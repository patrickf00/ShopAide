The client, \textbf{\emph{Tim Dockins}}, is an alumnus of the University of Texas at Arlington. Tim was trying to buy ingredients for his weekly meals and after going to a couple of stores, he still could not buy all of the ingredients he needed. That was when Tim realized that it would be very helpful if there is a groceries shopping application that can tell him where he can get what he needs with the most savings and shortest trips. 

Currently, Tim uses the Alexa application to keep his shopping list but he does not use any shopping application to help with comparing prices or distances. Tim acknowledged that many shopping applications help to find sales and coupons; however, no application can combine all aspects and suggest the optimal choice regarding both savings and distance. Therefore, this is a good opportunity for the team to build an application that can solve the problem, gain users and bring value to the community. 

The client does not have a preference for technical requirements such as a platform or database. However, the client wants to be able to access the application on multiple platforms. The application will allow manual entry to search for items and have a shopping list component. A user will have an account to save his or her shopping list and other settings. The application can access multiple stores and has a map component to navigate users to the stores. The client does not have a preference for the favorite stores, any nearby stores that have the items are accepted. Searching items by brands is not the client's priority. The desired component is to have in-application recipes to pick from or recipes imported from another website. Regarding the documentation and manual, Tim prefers the in-application manual and wants to have access to source code and documents but does not have a preference of how they should be done.
