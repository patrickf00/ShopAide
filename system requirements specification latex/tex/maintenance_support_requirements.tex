%Include a header paragraph specific to your product here. Maintenance and support requirements address items specific to the ongoing maintenance and support of your product after delivery. Think of these requirements as if you were the ones who would be responsible for caring for customers/end user after the product is delivered in its final form and in use "in the field". What would you require to do this job? Specify items such as: where, how and who must be able to maintain the product to correct errors, hardware failures, etc.; required support/troubleshooting manuals/guides; availability/documentation of source code; related technical documentation that must be available for maintainers; specific/unique tools required for maintenance; specific software/environment required for maintenance; etc.
For maintenance on this project, the team at present has no expectation for continuing support on this project after its completion.  The following are a list of deliverables and notes for the support team, which the customer may choose to appoint at a later date.

\subsection{Source Code Documentation}
\subsubsection{Description}
The documentation within and possibly extracted from the source code, which will define how each class and public method behaves.
\subsubsection{Source}
UTA STeam
\subsubsection{Constraints}
Can only be implemented as time allows for this project and given the development process.
%\subsubsection{Standards}
%List of applicable standards
\subsubsection{Priority}
Low

\subsection{Heroku Maintenance}
\subsubsection{Description}
For the application to remain active and usable, Heroku must remain running.
\subsubsection{Source}
UTA STeam
\subsubsection{Constraints}
If no activity is performed on the website or on the Heroku Interface, it will be deleted by the system.
%\subsubsection{Standards}
%List of applicable standards
\subsubsection{Priority}
Low
